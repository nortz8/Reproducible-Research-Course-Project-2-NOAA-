\documentclass[]{article}
\usepackage{lmodern}
\usepackage{amssymb,amsmath}
\usepackage{ifxetex,ifluatex}
\usepackage{fixltx2e} % provides \textsubscript
\ifnum 0\ifxetex 1\fi\ifluatex 1\fi=0 % if pdftex
  \usepackage[T1]{fontenc}
  \usepackage[utf8]{inputenc}
\else % if luatex or xelatex
  \ifxetex
    \usepackage{mathspec}
  \else
    \usepackage{fontspec}
  \fi
  \defaultfontfeatures{Ligatures=TeX,Scale=MatchLowercase}
\fi
% use upquote if available, for straight quotes in verbatim environments
\IfFileExists{upquote.sty}{\usepackage{upquote}}{}
% use microtype if available
\IfFileExists{microtype.sty}{%
\usepackage{microtype}
\UseMicrotypeSet[protrusion]{basicmath} % disable protrusion for tt fonts
}{}
\usepackage[margin=1in]{geometry}
\usepackage{hyperref}
\hypersetup{unicode=true,
            pdftitle={Top Natural Disaster Events in terms of Economic and Population Health Damages in the United States from 1950 to 2011},
            pdfauthor={Gerald Del Norte},
            pdfborder={0 0 0},
            breaklinks=true}
\urlstyle{same}  % don't use monospace font for urls
\usepackage{color}
\usepackage{fancyvrb}
\newcommand{\VerbBar}{|}
\newcommand{\VERB}{\Verb[commandchars=\\\{\}]}
\DefineVerbatimEnvironment{Highlighting}{Verbatim}{commandchars=\\\{\}}
% Add ',fontsize=\small' for more characters per line
\usepackage{framed}
\definecolor{shadecolor}{RGB}{248,248,248}
\newenvironment{Shaded}{\begin{snugshade}}{\end{snugshade}}
\newcommand{\KeywordTok}[1]{\textcolor[rgb]{0.13,0.29,0.53}{\textbf{#1}}}
\newcommand{\DataTypeTok}[1]{\textcolor[rgb]{0.13,0.29,0.53}{#1}}
\newcommand{\DecValTok}[1]{\textcolor[rgb]{0.00,0.00,0.81}{#1}}
\newcommand{\BaseNTok}[1]{\textcolor[rgb]{0.00,0.00,0.81}{#1}}
\newcommand{\FloatTok}[1]{\textcolor[rgb]{0.00,0.00,0.81}{#1}}
\newcommand{\ConstantTok}[1]{\textcolor[rgb]{0.00,0.00,0.00}{#1}}
\newcommand{\CharTok}[1]{\textcolor[rgb]{0.31,0.60,0.02}{#1}}
\newcommand{\SpecialCharTok}[1]{\textcolor[rgb]{0.00,0.00,0.00}{#1}}
\newcommand{\StringTok}[1]{\textcolor[rgb]{0.31,0.60,0.02}{#1}}
\newcommand{\VerbatimStringTok}[1]{\textcolor[rgb]{0.31,0.60,0.02}{#1}}
\newcommand{\SpecialStringTok}[1]{\textcolor[rgb]{0.31,0.60,0.02}{#1}}
\newcommand{\ImportTok}[1]{#1}
\newcommand{\CommentTok}[1]{\textcolor[rgb]{0.56,0.35,0.01}{\textit{#1}}}
\newcommand{\DocumentationTok}[1]{\textcolor[rgb]{0.56,0.35,0.01}{\textbf{\textit{#1}}}}
\newcommand{\AnnotationTok}[1]{\textcolor[rgb]{0.56,0.35,0.01}{\textbf{\textit{#1}}}}
\newcommand{\CommentVarTok}[1]{\textcolor[rgb]{0.56,0.35,0.01}{\textbf{\textit{#1}}}}
\newcommand{\OtherTok}[1]{\textcolor[rgb]{0.56,0.35,0.01}{#1}}
\newcommand{\FunctionTok}[1]{\textcolor[rgb]{0.00,0.00,0.00}{#1}}
\newcommand{\VariableTok}[1]{\textcolor[rgb]{0.00,0.00,0.00}{#1}}
\newcommand{\ControlFlowTok}[1]{\textcolor[rgb]{0.13,0.29,0.53}{\textbf{#1}}}
\newcommand{\OperatorTok}[1]{\textcolor[rgb]{0.81,0.36,0.00}{\textbf{#1}}}
\newcommand{\BuiltInTok}[1]{#1}
\newcommand{\ExtensionTok}[1]{#1}
\newcommand{\PreprocessorTok}[1]{\textcolor[rgb]{0.56,0.35,0.01}{\textit{#1}}}
\newcommand{\AttributeTok}[1]{\textcolor[rgb]{0.77,0.63,0.00}{#1}}
\newcommand{\RegionMarkerTok}[1]{#1}
\newcommand{\InformationTok}[1]{\textcolor[rgb]{0.56,0.35,0.01}{\textbf{\textit{#1}}}}
\newcommand{\WarningTok}[1]{\textcolor[rgb]{0.56,0.35,0.01}{\textbf{\textit{#1}}}}
\newcommand{\AlertTok}[1]{\textcolor[rgb]{0.94,0.16,0.16}{#1}}
\newcommand{\ErrorTok}[1]{\textcolor[rgb]{0.64,0.00,0.00}{\textbf{#1}}}
\newcommand{\NormalTok}[1]{#1}
\usepackage{graphicx,grffile}
\makeatletter
\def\maxwidth{\ifdim\Gin@nat@width>\linewidth\linewidth\else\Gin@nat@width\fi}
\def\maxheight{\ifdim\Gin@nat@height>\textheight\textheight\else\Gin@nat@height\fi}
\makeatother
% Scale images if necessary, so that they will not overflow the page
% margins by default, and it is still possible to overwrite the defaults
% using explicit options in \includegraphics[width, height, ...]{}
\setkeys{Gin}{width=\maxwidth,height=\maxheight,keepaspectratio}
\IfFileExists{parskip.sty}{%
\usepackage{parskip}
}{% else
\setlength{\parindent}{0pt}
\setlength{\parskip}{6pt plus 2pt minus 1pt}
}
\setlength{\emergencystretch}{3em}  % prevent overfull lines
\providecommand{\tightlist}{%
  \setlength{\itemsep}{0pt}\setlength{\parskip}{0pt}}
\setcounter{secnumdepth}{0}
% Redefines (sub)paragraphs to behave more like sections
\ifx\paragraph\undefined\else
\let\oldparagraph\paragraph
\renewcommand{\paragraph}[1]{\oldparagraph{#1}\mbox{}}
\fi
\ifx\subparagraph\undefined\else
\let\oldsubparagraph\subparagraph
\renewcommand{\subparagraph}[1]{\oldsubparagraph{#1}\mbox{}}
\fi

%%% Use protect on footnotes to avoid problems with footnotes in titles
\let\rmarkdownfootnote\footnote%
\def\footnote{\protect\rmarkdownfootnote}

%%% Change title format to be more compact
\usepackage{titling}

% Create subtitle command for use in maketitle
\newcommand{\subtitle}[1]{
  \posttitle{
    \begin{center}\large#1\end{center}
    }
}

\setlength{\droptitle}{-2em}

  \title{Top Natural Disaster Events in terms of Economic and Population Health
Damages in the United States from 1950 to 2011}
    \pretitle{\vspace{\droptitle}\centering\huge}
  \posttitle{\par}
    \author{Gerald Del Norte}
    \preauthor{\centering\large\emph}
  \postauthor{\par}
      \predate{\centering\large\emph}
  \postdate{\par}
    \date{February 9, 2019}


\begin{document}
\maketitle

\section{Top Natural Disaster Events in terms of Economic and Population
Health Damages in the United States from 1950 to
2011}\label{top-natural-disaster-events-in-terms-of-economic-and-population-health-damages-in-the-united-states-from-1950-to-2011}

\subsection{Synopsis}\label{synopsis}

Storms and other severe weather events can cause both public health and
economic problems for communities and municipalities. Many severe events
can result in fatalities, injuries, and property damage, and preventing
such outcomes to the extent possible is a key concern.

This document involves exploring the U.S. National Oceanic and
Atmospheric Administration's (NOAA) storm database. This database tracks
characteristics of major storms and weather events in the United States,
including when and where they occur, as well as estimates of any
fatalities, injuries, and property damage.

\subsection{Data Processing (Downloading, Loading and Cleaning
Data)}\label{data-processing-downloading-loading-and-cleaning-data}

\begin{Shaded}
\begin{Highlighting}[]
\NormalTok{## installs and loads packages(if not yet installed)}
\ControlFlowTok{if}\NormalTok{(}\OperatorTok{!}\KeywordTok{require}\NormalTok{(png))\{}
  \KeywordTok{install.packages}\NormalTok{(}\StringTok{"png"}\NormalTok{)}
  \KeywordTok{library}\NormalTok{(png)}
\NormalTok{\}}
\end{Highlighting}
\end{Shaded}

\begin{verbatim}
## Loading required package: png
\end{verbatim}

\begin{Shaded}
\begin{Highlighting}[]
\ControlFlowTok{if}\NormalTok{(}\OperatorTok{!}\KeywordTok{require}\NormalTok{(plyr))\{}
  \KeywordTok{install.packages}\NormalTok{(}\StringTok{"plyr"}\NormalTok{)}
  \KeywordTok{library}\NormalTok{(plyr)}
\NormalTok{\}}
\end{Highlighting}
\end{Shaded}

\begin{verbatim}
## Loading required package: plyr
\end{verbatim}

\begin{Shaded}
\begin{Highlighting}[]
\ControlFlowTok{if}\NormalTok{(}\OperatorTok{!}\KeywordTok{require}\NormalTok{(dplyr))\{}
  \KeywordTok{install.packages}\NormalTok{(}\StringTok{"dplyr"}\NormalTok{)}
  \KeywordTok{library}\NormalTok{(dplyr)}
\NormalTok{\}}
\end{Highlighting}
\end{Shaded}

\begin{verbatim}
## Loading required package: dplyr
\end{verbatim}

\begin{verbatim}
## 
## Attaching package: 'dplyr'
\end{verbatim}

\begin{verbatim}
## The following objects are masked from 'package:plyr':
## 
##     arrange, count, desc, failwith, id, mutate, rename, summarise,
##     summarize
\end{verbatim}

\begin{verbatim}
## The following objects are masked from 'package:stats':
## 
##     filter, lag
\end{verbatim}

\begin{verbatim}
## The following objects are masked from 'package:base':
## 
##     intersect, setdiff, setequal, union
\end{verbatim}

\begin{Shaded}
\begin{Highlighting}[]
\ControlFlowTok{if}\NormalTok{(}\OperatorTok{!}\KeywordTok{require}\NormalTok{(lattice))\{}
  \KeywordTok{install.packages}\NormalTok{(}\StringTok{"lattice"}\NormalTok{)}
  \KeywordTok{library}\NormalTok{(lattice)}
\NormalTok{\}}
\end{Highlighting}
\end{Shaded}

\begin{verbatim}
## Loading required package: lattice
\end{verbatim}

\begin{Shaded}
\begin{Highlighting}[]
\KeywordTok{library}\NormalTok{(png)}
\KeywordTok{library}\NormalTok{(plyr)}
\KeywordTok{library}\NormalTok{(dplyr)}
\KeywordTok{library}\NormalTok{(lattice) }

\NormalTok{### Downloads the source file if it is not in the current working directory, proceeds if it is }

\NormalTok{destfilebz2 =}\StringTok{ "./repdata_data_StormData.csv.bz2"}
\NormalTok{fileURL <-}\StringTok{ "https://d396qusza40orc.cloudfront.net/repdata%2Fdata%2FStormData.csv.bz2"}
\ControlFlowTok{if}\NormalTok{(}\OperatorTok{!}\KeywordTok{file.exists}\NormalTok{(destfilebz2))\{}
  \KeywordTok{download.file}\NormalTok{(fileURL, }\DataTypeTok{destfile=}\StringTok{"./repdata_data_StormData.csv.bz2"}\NormalTok{, }\DataTypeTok{method=}\StringTok{"auto"}\NormalTok{)}
\NormalTok{\}}

\CommentTok{#Reads and cleans data}
  
\NormalTok{stormx <-}\StringTok{ }\KeywordTok{read.csv}\NormalTok{(}\StringTok{"./repdata_data_StormData.csv.bz2"}\NormalTok{)}
  
\NormalTok{Statelist<-}\StringTok{ }\KeywordTok{list}\NormalTok{(}\StringTok{"AK"}\NormalTok{,}\StringTok{"AL"}\NormalTok{,}\StringTok{"AR"}\NormalTok{,}\StringTok{"AZ"}\NormalTok{,}\StringTok{"CA"}\NormalTok{,}\StringTok{"CO"}\NormalTok{,}\StringTok{"CT"}\NormalTok{,}\StringTok{"DC"}\NormalTok{,}\StringTok{"DE"}\NormalTok{,}\StringTok{"FL"}\NormalTok{,}\StringTok{"GA"}\NormalTok{,}\StringTok{"GU"}\NormalTok{,}\StringTok{"HI"}\NormalTok{,}\StringTok{"IA"}\NormalTok{,}\StringTok{"ID"}\NormalTok{, }\StringTok{"IL"}\NormalTok{,}\StringTok{"IN"}\NormalTok{,}\StringTok{"KS"}\NormalTok{,}\StringTok{"KY"}\NormalTok{,}\StringTok{"LA"}\NormalTok{,}\StringTok{"MA"}\NormalTok{,}\StringTok{"MD"}\NormalTok{,}\StringTok{"ME"}\NormalTok{,}\StringTok{"MH"}\NormalTok{,}\StringTok{"MI"}\NormalTok{,}\StringTok{"MN"}\NormalTok{,}\StringTok{"MO"}\NormalTok{,}\StringTok{"MS"}\NormalTok{,}\StringTok{"MT"}\NormalTok{,}\StringTok{"NC"}\NormalTok{,}\StringTok{"ND"}\NormalTok{,}\StringTok{"NE"}\NormalTok{,}\StringTok{"NH"}\NormalTok{,}\StringTok{"NJ"}\NormalTok{,}\StringTok{"NM"}\NormalTok{,}\StringTok{"NV"}\NormalTok{,}\StringTok{"NY"}\NormalTok{, }\StringTok{"OH"}\NormalTok{,}\StringTok{"OK"}\NormalTok{,}\StringTok{"OR"}\NormalTok{,}\StringTok{"PA"}\NormalTok{,}\StringTok{"PR"}\NormalTok{,}\StringTok{"PW"}\NormalTok{,}\StringTok{"RI"}\NormalTok{,}\StringTok{"SC"}\NormalTok{,}\StringTok{"SD"}\NormalTok{,}\StringTok{"TN"}\NormalTok{,}\StringTok{"TX"}\NormalTok{,}\StringTok{"UT"}\NormalTok{,}\StringTok{"VA"}\NormalTok{,}\StringTok{"VI"}\NormalTok{,}\StringTok{"VT"}\NormalTok{,}\StringTok{"WA"}\NormalTok{,}\StringTok{"WI"}\NormalTok{,}\StringTok{"WV"}\NormalTok{,}\StringTok{"WY"}\NormalTok{)}

\CommentTok{#Removes unrelated rows by only retaining the rows that have state in the STATE column.}

\NormalTok{storm <-}\StringTok{ }\NormalTok{stormx[ stormx}\OperatorTok{$}\NormalTok{STATE }\OperatorTok\StringTok{ }\NormalTok{Statelist,]}

\CommentTok{#Removes columns not needed for this analysis}

\NormalTok{storm <-}\StringTok{ }\KeywordTok{select}\NormalTok{(storm, EVTYPE, FATALITIES, INJURIES, PROPDMG, PROPDMGEXP, CROPDMG, CROPDMGEXP)}


\CommentTok{#This next section substitutes values on the PROPDMGEXP and CROPDMG EXP to make them numerical, source of information is from https://rpubs.com/flyingdisc/PROPDMGEXP}


\CommentTok{#creates a backup csv file and reads the file with strings as characters}
\KeywordTok{write.csv}\NormalTok{(storm,}\StringTok{'storm.csv'}\NormalTok{)}
\NormalTok{storm<-}\KeywordTok{read.csv}\NormalTok{(}\StringTok{"storm.csv"}\NormalTok{,}\DataTypeTok{row.names=}\OtherTok{NULL}\NormalTok{, }\DataTypeTok{stringsAsFactors =} \OtherTok{FALSE}\NormalTok{)}


\NormalTok{##creates replacement dataframe for PROPDMGEXP and CROPDMGEXP}
\NormalTok{PROPDMGEXP =}\StringTok{ }\KeywordTok{c}\NormalTok{(}\StringTok{"H"}\NormalTok{,}\StringTok{"h"}\NormalTok{,}\StringTok{"K"}\NormalTok{,}\StringTok{"k"}\NormalTok{,}\StringTok{"M"}\NormalTok{,}\StringTok{"m"}\NormalTok{,}\StringTok{"B"}\NormalTok{,}\StringTok{"b"}\NormalTok{,}\StringTok{"+"}\NormalTok{,}\StringTok{"-"}\NormalTok{,}\StringTok{"?"}\NormalTok{,}\StringTok{"1"}\NormalTok{,}\StringTok{"2"}\NormalTok{,}\StringTok{"3"}\NormalTok{,}\StringTok{"4"}\NormalTok{,}\StringTok{"5"}\NormalTok{,}\StringTok{"6"}\NormalTok{,}\StringTok{"7"}\NormalTok{,}\StringTok{"8"}\NormalTok{,}\StringTok{"0"}\NormalTok{,}\StringTok{""}\NormalTok{) }
\NormalTok{PROPREPVAL =}\StringTok{ }\KeywordTok{c}\NormalTok{(}\StringTok{"100"}\NormalTok{,}\StringTok{"100"}\NormalTok{,}\StringTok{"1000"}\NormalTok{,}\StringTok{"1000"}\NormalTok{,}\StringTok{"1000000"}\NormalTok{,}\StringTok{"1000000"}\NormalTok{,}\StringTok{"1000000000"}\NormalTok{,}\StringTok{"1000000000"}\NormalTok{,}\StringTok{"1"}\NormalTok{,}\StringTok{"0"}\NormalTok{,}\StringTok{"0"}\NormalTok{,}\StringTok{"10"}\NormalTok{,}\StringTok{"10"}\NormalTok{,}\StringTok{"10"}\NormalTok{,}\StringTok{"10"}\NormalTok{,}\StringTok{"10"}\NormalTok{,}\StringTok{"10"}\NormalTok{,}\StringTok{"10"}\NormalTok{,}\StringTok{"10"}\NormalTok{,}\StringTok{"10"}\NormalTok{,}\StringTok{"0"}\NormalTok{)}

\NormalTok{CROPDMGEXP =}\StringTok{ }\KeywordTok{c}\NormalTok{(}\StringTok{"H"}\NormalTok{,}\StringTok{"h"}\NormalTok{,}\StringTok{"K"}\NormalTok{,}\StringTok{"k"}\NormalTok{,}\StringTok{"M"}\NormalTok{,}\StringTok{"m"}\NormalTok{,}\StringTok{"B"}\NormalTok{,}\StringTok{"b"}\NormalTok{,}\StringTok{"+"}\NormalTok{,}\StringTok{"-"}\NormalTok{,}\StringTok{"?"}\NormalTok{,}\StringTok{"1"}\NormalTok{,}\StringTok{"2"}\NormalTok{,}\StringTok{"3"}\NormalTok{,}\StringTok{"4"}\NormalTok{,}\StringTok{"5"}\NormalTok{,}\StringTok{"6"}\NormalTok{,}\StringTok{"7"}\NormalTok{,}\StringTok{"8"}\NormalTok{,}\StringTok{"0"}\NormalTok{,}\StringTok{""}\NormalTok{) }
\NormalTok{CROPREPVAL =}\StringTok{ }\KeywordTok{c}\NormalTok{(}\StringTok{"100"}\NormalTok{,}\StringTok{"100"}\NormalTok{,}\StringTok{"1000"}\NormalTok{,}\StringTok{"1000"}\NormalTok{,}\StringTok{"1000000"}\NormalTok{,}\StringTok{"1000000"}\NormalTok{,}\StringTok{"1000000000"}\NormalTok{,}\StringTok{"1000000000"}\NormalTok{,}\StringTok{"1"}\NormalTok{,}\StringTok{"0"}\NormalTok{,}\StringTok{"0"}\NormalTok{,}\StringTok{"10"}\NormalTok{,}\StringTok{"10"}\NormalTok{,}\StringTok{"10"}\NormalTok{,}\StringTok{"10"}\NormalTok{,}\StringTok{"10"}\NormalTok{,}\StringTok{"10"}\NormalTok{,}\StringTok{"10"}\NormalTok{,}\StringTok{"10"}\NormalTok{,}\StringTok{"10"}\NormalTok{,}\StringTok{"0"}\NormalTok{)}


\CommentTok{#merges the new replacement values with the original dataframe}
\NormalTok{propexpdf =}\StringTok{ }\KeywordTok{data.frame}\NormalTok{(PROPDMGEXP, PROPREPVAL)}
\NormalTok{storm2 <-}\StringTok{ }\KeywordTok{merge}\NormalTok{(}\DataTypeTok{x=}\NormalTok{storm, }\DataTypeTok{y=}\NormalTok{propexpdf, }\DataTypeTok{by =} \StringTok{"PROPDMGEXP"}\NormalTok{, }\DataTypeTok{all.x=}\OtherTok{TRUE}\NormalTok{)}

\NormalTok{cropexpdf =}\StringTok{ }\KeywordTok{data.frame}\NormalTok{(CROPDMGEXP, CROPREPVAL)}
\NormalTok{storm3 <-}\StringTok{ }\KeywordTok{merge}\NormalTok{(}\DataTypeTok{x=}\NormalTok{storm2, }\DataTypeTok{y=}\NormalTok{cropexpdf, }\DataTypeTok{by =} \StringTok{"CROPDMGEXP"}\NormalTok{, }\DataTypeTok{all.x=}\OtherTok{TRUE}\NormalTok{)}

\CommentTok{#creates another backup prior to plotting}
\KeywordTok{write.csv}\NormalTok{(storm3,}\StringTok{'storm3.csv'}\NormalTok{)}
\NormalTok{storm3<-}\KeywordTok{read.csv}\NormalTok{(}\StringTok{"storm3.csv"}\NormalTok{,}\DataTypeTok{row.names=}\OtherTok{NULL}\NormalTok{, }\DataTypeTok{stringsAsFactors =} \OtherTok{TRUE}\NormalTok{)}
\end{Highlighting}
\end{Shaded}

\subsection{Results}\label{results}

\begin{Shaded}
\begin{Highlighting}[]
\CommentTok{# Aggregate and summarizes the top the 5 events with most fatalities}

\NormalTok{  deaths <-}\StringTok{ }\KeywordTok{aggregate}\NormalTok{(FATALITIES}\OperatorTok{~}\NormalTok{EVTYPE, storm3, sum)}
\NormalTok{  deaths <-}\StringTok{ }\NormalTok{deaths[}\KeywordTok{with}\NormalTok{(deaths, }\KeywordTok{order}\NormalTok{(}\OperatorTok{-}\NormalTok{FATALITIES)), ]}
\NormalTok{  deaths <-}\StringTok{ }\NormalTok{deaths[}\DecValTok{1}\OperatorTok{:}\DecValTok{5}\NormalTok{,]}
  \KeywordTok{summary}\NormalTok{(deaths)}
\end{Highlighting}
\end{Shaded}

\begin{verbatim}
##             EVTYPE    FATALITIES  
##  EXCESSIVE HEAT:1   Min.   : 815  
##  FLASH FLOOD   :1   1st Qu.: 937  
##  HEAT          :1   Median : 974  
##  LIGHTNING     :1   Mean   :2052  
##  TORNADO       :1   3rd Qu.:1903  
##   COASTAL FLOOD:0   Max.   :5633  
##  (Other)       :0
\end{verbatim}

\begin{Shaded}
\begin{Highlighting}[]
  \KeywordTok{head}\NormalTok{(deaths)}
\end{Highlighting}
\end{Shaded}

\begin{verbatim}
##             EVTYPE FATALITIES
## 826        TORNADO       5633
## 129 EXCESSIVE HEAT       1903
## 152    FLASH FLOOD        974
## 274           HEAT        937
## 461      LIGHTNING        815
\end{verbatim}

\begin{Shaded}
\begin{Highlighting}[]
  \KeywordTok{png}\NormalTok{(}\StringTok{"plot1.png"}\NormalTok{, }\DataTypeTok{width=}\DecValTok{800}\NormalTok{, }\DataTypeTok{height=}\DecValTok{600}\NormalTok{)}
\NormalTok{  plot1 <-}\StringTok{ }\KeywordTok{barplot}\NormalTok{(deaths}\OperatorTok{$}\NormalTok{FATALITIES, }\DataTypeTok{names =}\NormalTok{ deaths}\OperatorTok{$}\NormalTok{EVTYPE, }\DataTypeTok{xlab =} \StringTok{"Events"}\NormalTok{, }\DataTypeTok{ylab =} \StringTok{"Fatalities"}\NormalTok{, }\DataTypeTok{main =} \StringTok{"Deaths by Event Type"}\NormalTok{)}
  \KeywordTok{dev.off}\NormalTok{() }
\end{Highlighting}
\end{Shaded}

\begin{verbatim}
## pdf 
##   2
\end{verbatim}

\begin{Shaded}
\begin{Highlighting}[]
\NormalTok{  img1 <-}\StringTok{ }\KeywordTok{readPNG}\NormalTok{(}\StringTok{"./plot1.png"}\NormalTok{)}
\NormalTok{  grid}\OperatorTok{::}\KeywordTok{grid.raster}\NormalTok{(img1)}
  
\CommentTok{# Aggregate and summarizes the top the 5 events with most injuries  }
  
\NormalTok{  injur <-}\StringTok{ }\KeywordTok{aggregate}\NormalTok{(INJURIES}\OperatorTok{~}\NormalTok{EVTYPE, storm3, sum)}
\NormalTok{  injur <-}\StringTok{ }\NormalTok{injur[}\KeywordTok{with}\NormalTok{(injur, }\KeywordTok{order}\NormalTok{(}\OperatorTok{-}\NormalTok{INJURIES)), ]}
\NormalTok{  injur <-}\StringTok{ }\NormalTok{injur[}\DecValTok{1}\OperatorTok{:}\DecValTok{5}\NormalTok{,]}
  \KeywordTok{summary}\NormalTok{(injur)}
\end{Highlighting}
\end{Shaded}

\begin{verbatim}
##             EVTYPE     INJURIES    
##  EXCESSIVE HEAT:1   Min.   : 5229  
##  FLOOD         :1   1st Qu.: 6525  
##  LIGHTNING     :1   Median : 6789  
##  TORNADO       :1   Mean   :23369  
##  TSTM WIND     :1   3rd Qu.: 6957  
##   COASTAL FLOOD:0   Max.   :91346  
##  (Other)       :0
\end{verbatim}

\begin{Shaded}
\begin{Highlighting}[]
  \KeywordTok{head}\NormalTok{(injur)}
\end{Highlighting}
\end{Shaded}

\begin{verbatim}
##             EVTYPE INJURIES
## 826        TORNADO    91346
## 848      TSTM WIND     6957
## 169          FLOOD     6789
## 129 EXCESSIVE HEAT     6525
## 461      LIGHTNING     5229
\end{verbatim}

\begin{Shaded}
\begin{Highlighting}[]
  \KeywordTok{png}\NormalTok{(}\StringTok{"plot2.png"}\NormalTok{, }\DataTypeTok{width=}\DecValTok{800}\NormalTok{, }\DataTypeTok{height=}\DecValTok{600}\NormalTok{)}
\NormalTok{  plot2 <-}\StringTok{ }\KeywordTok{barplot}\NormalTok{(injur}\OperatorTok{$}\NormalTok{INJURIES, }\DataTypeTok{names =}\NormalTok{ injur}\OperatorTok{$}\NormalTok{EVTYPE, }\DataTypeTok{xlab =} \StringTok{"Events"}\NormalTok{, }\DataTypeTok{ylab =} \StringTok{"Injuries"}\NormalTok{, }\DataTypeTok{main =} \StringTok{"Injuries by Event Type"}\NormalTok{)}
  \KeywordTok{dev.off}\NormalTok{() }
\end{Highlighting}
\end{Shaded}

\begin{verbatim}
## pdf 
##   2
\end{verbatim}

\begin{Shaded}
\begin{Highlighting}[]
\NormalTok{  img2 <-}\StringTok{ }\KeywordTok{readPNG}\NormalTok{(}\StringTok{"./plot2.png"}\NormalTok{)}
\NormalTok{  grid}\OperatorTok{::}\KeywordTok{grid.raster}\NormalTok{(img2)}
  
\CommentTok{# Set factors as numeric  }
\NormalTok{storm3[,}\StringTok{"CROPDMG"}\NormalTok{] <-}\StringTok{ }\KeywordTok{as.numeric}\NormalTok{(}\KeywordTok{as.character}\NormalTok{(storm3[,}\StringTok{"CROPDMG"}\NormalTok{]))}
\NormalTok{storm3[,}\StringTok{"CROPREPVAL"}\NormalTok{] <-}\StringTok{ }\KeywordTok{as.numeric}\NormalTok{(}\KeywordTok{as.character}\NormalTok{(storm3[,}\StringTok{"CROPREPVAL"}\NormalTok{]))}
\NormalTok{storm3[,}\StringTok{"PROPDMG"}\NormalTok{] <-}\StringTok{ }\KeywordTok{as.numeric}\NormalTok{(}\KeywordTok{as.character}\NormalTok{(storm3[,}\StringTok{"PROPDMG"}\NormalTok{]))}
\NormalTok{storm3[,}\StringTok{"PROPREPVAL"}\NormalTok{] <-}\StringTok{ }\KeywordTok{as.numeric}\NormalTok{(}\KeywordTok{as.character}\NormalTok{(storm3[,}\StringTok{"PROPREPVAL"}\NormalTok{]))}

\CommentTok{# Mutates the dataframe by using the exponents for crop and property damage, then adds the two to determine economic cost. }
\NormalTok{storm4 <-}\StringTok{ }\KeywordTok{mutate}\NormalTok{(storm3, }\DataTypeTok{CROPTOTAL =}\NormalTok{ CROPREPVAL }\OperatorTok{*}\StringTok{ }\NormalTok{CROPDMG)}
\NormalTok{storm5 <-}\StringTok{ }\KeywordTok{mutate}\NormalTok{(storm4, }\DataTypeTok{PROPTOTAL =}\NormalTok{ PROPREPVAL }\OperatorTok{*}\StringTok{ }\NormalTok{PROPDMG)}
\NormalTok{storm6 <-}\StringTok{ }\KeywordTok{mutate}\NormalTok{(storm5, }\DataTypeTok{COSTTOTAL =}\NormalTok{ CROPTOTAL }\OperatorTok{+}\StringTok{ }\NormalTok{PROPTOTAL)}

\NormalTok{cost <-}\StringTok{ }\KeywordTok{aggregate}\NormalTok{(COSTTOTAL}\OperatorTok{~}\NormalTok{EVTYPE, storm6, sum)}
\NormalTok{cost <-}\StringTok{ }\NormalTok{cost[}\KeywordTok{with}\NormalTok{(cost, }\KeywordTok{order}\NormalTok{(}\OperatorTok{-}\NormalTok{COSTTOTAL)), ]}
\NormalTok{cost <-}\StringTok{ }\NormalTok{cost[}\DecValTok{1}\OperatorTok{:}\DecValTok{5}\NormalTok{,]}
\KeywordTok{summary}\NormalTok{(cost)}
\end{Highlighting}
\end{Shaded}

\begin{verbatim}
##                EVTYPE    COSTTOTAL        
##  FLOOD            :1   Min.   :1.876e+10  
##  HAIL             :1   1st Qu.:4.332e+10  
##  HURRICANE/TYPHOON:1   Median :5.735e+10  
##  STORM SURGE      :1   Mean   :6.832e+10  
##  TORNADO          :1   3rd Qu.:7.185e+10  
##   COASTAL FLOOD   :0   Max.   :1.503e+11  
##  (Other)          :0
\end{verbatim}

\begin{Shaded}
\begin{Highlighting}[]
\KeywordTok{head}\NormalTok{(cost)}
\end{Highlighting}
\end{Shaded}

\begin{verbatim}
##                EVTYPE    COSTTOTAL
## 169             FLOOD 150319131250
## 408 HURRICANE/TYPHOON  71853560800
## 826           TORNADO  57352117607
## 662       STORM SURGE  43323541000
## 243              HAIL  18758224527
\end{verbatim}

\begin{Shaded}
\begin{Highlighting}[]
\KeywordTok{png}\NormalTok{(}\StringTok{"plot3.png"}\NormalTok{, }\DataTypeTok{width=}\DecValTok{800}\NormalTok{, }\DataTypeTok{height=}\DecValTok{600}\NormalTok{)}
\NormalTok{plot3 <-}\StringTok{ }\KeywordTok{barplot}\NormalTok{(cost}\OperatorTok{$}\NormalTok{COSTTOTAL, }\DataTypeTok{names =}\NormalTok{ cost}\OperatorTok{$}\NormalTok{EVTYPE, }\DataTypeTok{xlab =} \StringTok{"Type of Event"}\NormalTok{, }\DataTypeTok{ylab =} \StringTok{"Total Economic Damage"}\NormalTok{, }\DataTypeTok{main =} \StringTok{"Total Damage Costs by Event Type"}\NormalTok{)}
\KeywordTok{dev.off}\NormalTok{() }
\end{Highlighting}
\end{Shaded}

\begin{verbatim}
## pdf 
##   2
\end{verbatim}

\begin{Shaded}
\begin{Highlighting}[]
\NormalTok{  img3 <-}\StringTok{ }\KeywordTok{readPNG}\NormalTok{(}\StringTok{"./plot3.png"}\NormalTok{)}
\NormalTok{  grid}\OperatorTok{::}\KeywordTok{grid.raster}\NormalTok{(img3)}
\end{Highlighting}
\end{Shaded}

\includegraphics{RepResCourseProj2_files/figure-latex/unnamed-chunk-2-1.pdf}


\end{document}
